// node.js的事件循环与浏览器有何区别
js是单线程执行的,也就是说一个一个进程里面只有一个主线程
多线程指的是一个程序里面有多个线程在同时执行不同的任务,允许单个程序同时创建多个并行执行的线程来完成各自的任务
浏览器内核是多线程,同时控制多个线程配相互配合以保持同步,
GUI线程
JavaScript引擎线程
定时触发器线程
事件触发线程
异步http请求

GUI线程主要负责页面的渲染,解析HTML、css,构建dom树,布局和绘制等
当界面需要重构时或者某种操作引发回流时,会触发该线程
该线程与JS引擎线程互斥,当执行js引擎线程时,GUI线程会被挂起,当任务队列清空时,JS引擎才会去执行GUI渲染

JS引擎主要负责处理JavaScript脚本

// web开发中js的设计模式
1.模块设计模式
指的就是js中的类,通常用于保持特殊代码和其他组件相互独立 ,互不干扰,确保类本身的行为和状态不被其他的类访问
2.揭示性模块模式
模块模式的另一种变称,主要是为了在保持封装性的同时,揭示在对象字面量中返回的变量和方法
3.原型设计模式
原型设计模式依赖于js的原型继承,原型模式主要用于为高性能环境创建对象
4.揭示性原型模式
类似于模块模式,原型模式也有一个揭示性模式。揭示性原型模式通过返回一个对象字面量,对公有和私有的成员进行封装。

//js常见的设计模式
1.单体模式,也叫单列模式
let data = {
  name: 'af',
  second() {
    console.log('second')
 }
}


